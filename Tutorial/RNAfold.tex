\documentclass{article}
\usepackage{graphicx}
\usepackage{hyperref}
\begin{document}

\section{RNA Folding Tutorial}

\subsection{Introduction}

For obvious reasons this tutorial focuses on the Vienna RNA Package.
In addition we'll use the RNA fold servers at these web sites:
\begin{itemize}
\item Michael Zuker's mfold server at\\
\url{http://www.ibc.wustl.edu/~zuker/rna/}
\item The European mfold server at\\
\url{http://bibiserv.techfak.uni-bielefeld.de/mfold/}
\item The Vienna RNA fold server at\\
\url{http://www.tbi.univie.ac.at/cgi-bin/RNAfold.cgi}
\end{itemize}

\subsection{Installing the Vienna RNA Package}

Unpack the tar file using {\tt zcat ViennaRNA-1.4.tar.gz | tar -xvf -}\\
Change into the {\tt ViennaRNA-1.4} directory you just created.
Type
\begin{quote}
\tt ./configure\\
make
\end{quote}
to build the library and the programs in the {\tt ./Progs}, {\tt ./Cluster}
and {\tt ./Utils} directories. You could now type
\begin{quote}
\tt make install
\end{quote}
if you want to install the executables for all to use.
See the files {\tt README} and {\tt INSTALL} for more detailed instructions. 

Add the {\tt Prog} to your {\tt PATH} environment variable and the {\tt man}
directory to the {\tt MANPATH} environment. For C-shell the commands would
be
\begin{quote}\tt
setenv PATH \$\{PATH\}:/wherever/ViennaRNA-1.5/Progs\\
setenv MANPATH \$\{MANPATH\}:/wherever/ViennaRNA-1.4/man
\end{quote}

You should now be able to execute the programs read the man pages. Take a
look at the {\tt RNAfold} man page by typing, {\tt man RNAfold}, lookup
all the other man pages before using each program.


\subsection{Structure Prediction}

Predict the structure of Yeast tRNA$^{\rm phe}$ sequence\\
{\small\tt
GCGGAUUUAGCUCAGUUGGGAGAGCGCCAGACUGAAGAUCUGGAGGUCCUGUGUUCGAUCCACAGAAUUCGCACCA}
using the above servers as well as the {\tt RNAfold} and {\tt RNAsubopt}
programs. The correct native structure can be seen in
figure~\ref{fig:tRNAsec}. 

\begin{figure}[ht]
\centerline{
\includegraphics[width=0.4\textwidth]{tRNA_phe_ss.ps}}
\caption{The native clover leaf structure of tRNA$^{\rm phe}$, modified
bases shown in bold.} 
\label{fig:tRNAsec}
\end{figure}

\begin{itemize}
\item How good is the predicted mfe structure?
\item Where (if at all) does the correct structure appear in the list of
suboptimal structures from {\tt mfold} and {\tt RNAsubopt}?
\item Look for the correct structure in the energy and pair probability
dot plots.
\item Test the effect of various options
\end{itemize}

Compute the energy of the clover leaf structure using {\tt RNAeval} and the
Zuker's {\tt efn} server. Estimate the frequency of the clover leaf
structure in thermodynamic equilibrium from its energy and the ensemble free
energy.

Interesting options for the Vienna programs {\tt RNAfold} and {\tt
RNAsubopt} are {\tt -d2}, {\tt -d3}, {\tt -noLP}, for {\tt mfold} vary the
window and percent suboptimality parameters.

Compute suboptimal structures and dot plots for the sequence\\
{\small\tt
GGGAGCGGUAGCUCCCGGAGAGCUAUUGCUCUCGUUGUAAUAAUGCAACGGCUGCAUUAUUGUAGC}\\
using {\tt mfold} and {\tt RNAsubopt}.
Why is the structure with two hairpins missing from the {\tt mfold}
output? Slowly increase the energy range in {\tt RNAsubopt} ({\tt -e}
option) and observe how fast the number of structures increases.

Note that the {\tt mfold} server in Bielefeld as well as the RNAfold server
in Vienna still use the older parameter set.

\subsubsection{Constrained folding}

Including experimental constraints can improve structure prediction
dramatically. The tRNA sequence above has several modified bases which are
unable to form base pairs, e.g.\ at positions 16,17,37,55.\\
Repeat the the predictions for the tRNA sequence, adding the constraint
that these positions cannot pair. Compare the results.

Try other constraints, such as forcing a particular base pair.

% add another example with chemical probing data

\subsubsection{Well-definedness}

Fold the sequence\\
{\small\tt GCUGCUGCUGCUGCUGCUGCUGCUGUUGCUGCGAGCCGAGCCCGUGAGGAGCCGAGCCGC}.
Energy and probability dot plots should show a well-defined structure on
the 3' half and many structural alternatives in the 5' half.

Create a mountain plot from the {\tt RNAfold -p} output using
\begin{quote}\tt
mountain.pl dot.ps
\end{quote} and plot the result. The first two curves show the
mountain representation of the mfe structure and base pair probabilities.
Differences in slope between these two indicate structural alternatives.
The third curve shows the positional entropy, a direct measure of local
well-definedness. 

Try the same for the tRNA structure folded with and without constraints.

For a longer example fold the RRE region of HIV, to be found in the file
{\tt RRE.seq}, and identify well- and ill-defined regions.

\subsection{Temperature dependence and melting curves}

Compute a melting curve of the tRNA sequence using the {\tt RNAheat}
program and plot it. Each peak of the specific heat corresponds to some
structural change. Fold the sequence at temperatures above and below each
peak to find the corresponding structures.

Note that the {\tt mfold} server will use the old parameter set for
temperatures other than 37C, since the current parameter set contains free
energies only. The Vienna RNA code mixes enthalpies from the old parameter
set with free energies from the current set.

\subsection{Inverse Folding}

Sequences with predefined structure can be designed by the Vienna RNA {\tt
RNAinverse} program as well as the inverse fold server at\\
\url{http://www.tbi.univie.ac.at/cgi-bin/RNAinverse.cgi}.\\
Lets try some examples.

Design a couple of sequences folding into the structures
\begin{quote}\tt
(((((((....))))))).......\\
..((......((((...)))).)).\\
\end{quote}

What does the result tell you about the frequency of these structures?

Design a few sequences folding into the tRNA clover leaf structure. Do
structure predictions for the resulting structures. Do the structures fold
correctly even when options are changes slightly? How well defined are the
structures?

Use the {\tt -Fmp} option to {\tt RNAinverse} and repeat the above
experiment; Compare results. 

The L-shaped tRNA tertiary structure is stabilized by a Watson Crick pair
pair between G19 and C56 and a another interaction G18 and U55. Design
sequences with these sequence constraints.

\subsection{Rolling your own}

The Vienna RNA package provides a {\tt C} library, as well a {\tt Perl5}
module, that facilitates developing your own programs. If time permits
try to implement a small program of your own, e.g.\ using {\tt Perl}.

Suggested problem: Write a {\tt Perl} script that reads in a sequence and
folds all subsequences form the 5' end to some position $k$, gradually
increasing $k$ up the full sequence length. The script could be useful in
identifying folding intermediates that appear when the incomplete sequence
folds during transcription. An example script re-implementing the {\tt
RNAfold} program can be found in the {\tt ./Perl} subdirectory, and may be
modified for this purpose.

\end{document}


% End of file
% LocalWords:  Zuker's mfold zcat xvf ViennaRNA Progs Utils PL MANPATH setenv
% LocalWords:  ATH ANPATH RNAfold RNAsubopt phe ss ps mfe RNAeval efn noLP pl
% LocalWords:  GCGGAUUUAGCUCAGUUGGGAGAGCGCCAGACUGAAGAUCUGGAGGUCCUGUGUUCGAUCCACAGAAUUCGCACCA
% LocalWords:  suboptimality Bielefeld definedness RNAheat enthalpies Fmp RRE
% LocalWords:  GCUGCUGCUGCUGCUGCUGCUGCUGUUGCUGCGAGCCGAGCCCGUGAGGAGCCGAGCCGC seq
% LocalWords:  RNAinverse
% LocalWords:  GGGAGCGGUAGCUCCCGGAGAGCUAUUGCUCUCGUUGUAAUAAUGCAACGGCUGCAUUAUUGUAGC
